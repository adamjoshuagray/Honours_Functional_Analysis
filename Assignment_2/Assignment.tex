\documentclass{unswmaths}

\begin{document}
\author{Adam J. Gray}
\title{Assignment}
\subject{Functional Analysis}
\studentno{3329798}

\unswtitle
\unswantiplagdec
\section*{Question 1}
See that
\begin{align*}
	\langle Tx, x \rangle &= \overline{\langle x, Tx \rangle} \\
		&\geq 0
\end{align*}
which means
\begin{align*}
	\overline{\langle x, Tx \rangle} &= \langle x, Tx \rangle \\
	    &= \langle T^* x, x \rangle
\end{align*}
so
\begin{align*}
	\langle (T - T^*)x, x \rangle &= 0
\end{align*}
for all $ x \in \mathcal{H} $.

We just want to show that $ (T - T^*) = 0 $. 
Say $ \langle Jx, x \rangle  = 0 $ for all $ x \in \mathcal{H} $ 
then for $ x, y \in \mathcal{H} $ see that 
\begin{align}
    \langle J(x + y), x + y \rangle &= \langle Jx, y \rangle 
        + \langle Jy, x \rangle + \underbrace{\langle Jx, x \rangle}_{=0}
        + \underbrace{\langle Jy, y \rangle}_{=0} \nonumber \\
        \label{eq:1}
        &= \langle Jx, y \rangle + \langle Jy, x \rangle
\end{align}
and see that
\begin{align} 
    \langle J(x + iy), x + iy \rangle &= \langle Jx, iy \rangle 
        + \langle Jiy, x \rangle + \underbrace{\langle Jx, x \rangle}_{=0}
        + \underbrace{\langle Jiy, iy \rangle}_{=0} \nonumber \\
        &= -i\langle Jx, y \rangle + i\langle Jy, x \rangle  
        \label{eq:2}
\end{align}
Now multiplying \eqref{eq:1} by $ i $ and subtracting it from \eqref{eq:2} yields
$$
    \langle Jx, y \rangle = 0
$$
for all $ x, y \in \mathcal{H} $ which implies $ J \equiv 0 $.
Letting $ J = (T - T^*) $, yields $ (T - T^*) = 0 $ as required.

So $ T \equiv T^* $. \qed

\section*{Question 3}
\subsubsection*{Part 1}
$ x \in (\operatorname{Im} T)^\perp $ iff
\begin{align*}
    & \langle Ty, x \rangle = 0 &\forall y \in \mathcal{H}\\
    & \Leftrightarrow \langle y, T^* x \rangle = 0 & \forall y \in \mathcal{H} \\
    & \Leftrightarrow T^* x = 0 \\
    & \Leftrightarrow x \in \ker T*
\end{align*}
So $ (\operatorname{Im} T)^\perp = \ker T^* $. \qed

\subsubsection*{Part 2}
Setting $ T = T^* $ in the result of \emph{part a} yields
\begin{align*}
    &(\operatorname{Im} T^*)^\perp = \ker T^{**}
    & \Leftrightarrow  (\operatorname{Im} T^*)^\perp = \ker T
\end{align*}
and taking orthogonal complements of both sides gets
\begin{align*}
    & (\operatorname{Im} T^*)^{\perp \perp} = (\ker T)^\perp \\
    & \Leftrightarrow \overline{\operatorname{Im} T^*} = (\ker T)^\perp.
\end{align*}
So $ \overline{\operatorname{Im} T^*} = (\ker T)^\perp $. \qed
\end{document}
