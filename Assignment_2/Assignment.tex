\documentclass{unswmaths}

\begin{document}
\author{Adam J. Gray}
\title{Assignment}
\subject{Functional Analysis}
\studentno{3329798}

\unswtitle
\unswantiplagdec
\section*{Question 1}
See that
\begin{align*}
	\langle Tx, x \rangle &= \overline{\langle x, Tx \rangle} \\
		&\geq 0
\end{align*}
which means
\begin{align*}
	\overline{\langle x, Tx \rangle} &= \langle x, Tx \rangle \\
	    &= \langle T^* x, x \rangle
\end{align*}
so
\begin{align*}
	\langle (T - T^*)x, x \rangle &= 0
\end{align*}
for all $ x \in \mathcal{H} $.

We just want to show that $ (T - T^*) = 0 $. 
Say $ \langle Jx, x \rangle  = 0 $ for all $ x \in \mathcal{H} $ 
then for $ x, y \in \mathcal{H} $ see that 
\begin{align}
    \langle J(x + y), x + y \rangle &= \langle Jx, y \rangle 
        + \langle Jy, x \rangle + \underbrace{\langle Jx, x \rangle}_{=0}
        + \underbrace{\langle Jy, y \rangle}_{=0} \nonumber \\
        \label{eq:1}
        &= \langle Jx, y \rangle + \langle Jy, x \rangle
\end{align}
and see that
\begin{align} 
    \langle J(x + iy), x + iy \rangle &= \langle Jx, iy \rangle 
        + \langle Jiy, x \rangle + \underbrace{\langle Jx, x \rangle}_{=0}
        + \underbrace{\langle Jiy, iy \rangle}_{=0} \nonumber \\
        &= -i\langle Jx, y \rangle + i\langle Jy, x \rangle  
        \label{eq:2}
\end{align}
Now multiplying \eqref{eq:1} by $ i $ and subtracting it from \eqref{eq:2} yields
$$
    \langle Jx, y \rangle = 0
$$
for all $ x, y \in \mathcal{H} $ which implies $ J \equiv 0 $.
Letting $ J = (T - T^*) $, yields $ (T - T^*) = 0 $ as required.

So $ T \equiv T^* $. \qed
\newpage
\section*{Question 2}
\subsubsection*{Part 1}
From a toeplitz operator $ T = \left\{ a{_i,j} \right\}_{i,j\in\mathbb{Z}} $
define the diagonal operator $ D_j $ as the matrix-type operator taken as then
$ j$-th diagonal of $ T $. Ie.
\begin{align*}
D_j = 
\left(
\begin{matrix}
	0 & \cdots & a_j & 0 & 0 & \cdots \\
	0 & \cdots & 0 & a_j & 0 & \cdots  \\
	0 & \cdots & 0 & 0 & \ddots & \cdots  \\
\end{matrix}
\right)
\end{align*}

If $ \mathbf{x} = (x_1, x_2, \ldots ) $ then it is clear that $ D_j \mathbf{x} = ( 0, \ldots, 0, a_j x_j, 0, \ldots ) $.
Also it is clear that 
$$
	T = \sum_{j \in \mathbb{Z}} D_j.
$$
Since we have that $ \sum_{j \in\mathbb{Z}} |a_j|  = M < \infty $ we can write
\begin{align*}
	||T\mathbf{x}|| &= || \sum_{j \in \mathbb{Z}} D_j \mathbf{x} || \\
		&\leq \sum_{j\in\mathbb{Z}} ||D_j \mathbf{x} || \\
		&\leq \sum_{j \in\mathbb{Z}} |a_j| ||\mathbf{x} || \\
		&= ||x|| \sum_{j \in\mathbb{Z}} |a_j| \\
		&= ||x|| M
\end{align*}
and so $ T $ is a bounded on $ \ell^2 $.

\subsubsection*{Part 3}
Using the idea that $ T = \sum_{j\in Z} D_j$ from part 1 we consider 
\begin{align*}
	\sum_{n=1}^{\infty} ||T e_n ||^2 &= \sum_{n=1}^{\infty} || \sum_{j \in \mathbb{Z}} D_j e_n ||^2 \\
		&= \sum_{n=1}^{\infty} || ( a_{n-1}, a_{n-2}, \ldots ) ||^2 \\
 		&= \sum_{n=1}^{\infty} || \sum_{j=0}^{\infty} a_{n-j}^2 || \\
\end{align*}

Now suppose that $ a_k \neq 0 $ for  some $ k $ then there exits an $ N $ such that $ n > N $
implies that $ a^2_k $ is in the sum $ \sum_{j=0}^{\infty} a_{n-j}^2 $ and consequently 
$ \sum_{n=1}^{\infty} || \sum_{j=0}^{\infty} a_{n-j}^2 || = \infty $.

So $ a_k =0 $ for all $ k $ and Hilbert-Schmidt Toeplitz operators are $ 0 $.

\newpage
\section*{Question 3}
\subsubsection*{Part 1}
$ x \in (\operatorname{Im} T)^\perp $ iff
\begin{align*}
    & \langle Ty, x \rangle = 0 &\forall y \in \mathcal{H}\\
    & \Leftrightarrow \langle y, T^* x \rangle = 0 & \forall y \in \mathcal{H} \\
    & \Leftrightarrow T^* x = 0 \\
    & \Leftrightarrow x \in \ker T*
\end{align*}
So $ (\operatorname{Im} T)^\perp = \ker T^* $. \qed

\subsubsection*{Part 2}
Setting $ T = T^* $ in the result of \emph{part a} yields
\begin{align*}
    &(\operatorname{Im} T^*)^\perp = \ker T^{**}
    & \Leftrightarrow  (\operatorname{Im} T^*)^\perp = \ker T
\end{align*}
and taking orthogonal complements of both sides gets
\begin{align*}
    & (\operatorname{Im} T^*)^{\perp \perp} = (\ker T)^\perp \\
    & \Leftrightarrow \overline{\operatorname{Im} T^*} = (\ker T)^\perp.
\end{align*}
So $ \overline{\operatorname{Im} T^*} = (\ker T)^\perp $. \qed
\end{document}
