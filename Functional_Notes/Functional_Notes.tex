\documentclass{unswmaths}
\usepackage{unswshortcuts}
\begin{document}
\author{Adam J. Gray}
\title{Lecture Notes}
\subject{Functional Analysis}
\studentno{3329798}

\unswtitle

\section*{Spectral Thoery}

Spectral theory is analogous to eigen-value theory for matricies.
In this case we look at spectral theory on bounded linear operators on 
a Hilbert space $ \Hlbt $. 

We define the following

For $ T \in B( \Hlbt ) $ we define the following
\begin{definition}[Resolvent Set]
	Define $ \rho(T) := \{ \lambda \in \Cplx : \exists (T-\lambda)^{-1}  \in B( \Hlbt ) \} $.
\end{definition}
\begin{definition}[Resolvent]
	Define $ R_\lambda(T) := (T - \lambda)^{-1} $
\end{definition}
\begin{definition}[Spectrum]
	Define $ \sigma(T) := \Cplx \setminus \rho(T) $.
\end{definition}

On could heuristically think of $ \lambda $ aas being an eigen-value and we formalize this idea as follows.

If $ \Hlbt = \Cplx^n $ and $ T = T_A $ where $ A = \left( a_{j,k} \right)_{j,k=1}^n $ (a matrix) then 
if $ (T - \lambda I)^{-1} $ does not exist, i.e. $ \ker( T - \lambda I) \neq \mathbf{0} $ then $ \lambda $ is
an eigen-value.

We would like to formalize and prove the following three ideas;
\begin{itemize}
	\item $ \sigma(T) $ is bounded,
	\item $ \sigma(T) $ is closed, and
	\item $ \sigma(T) $ is non-empty.
\end{itemize}

\begin{lemma}[$\sigma(T) $ is bounded]
\label{lem:sigma_bdd}
	If $ | \lambda | > ||T|| $ then $ \lambda \in \rho(T) $.
\end{lemma}
\begin{proof}
	If we have that $ | \lambda | < ||T|| $ then 
	\begin{align*}
		\sum_{k=0}^\infty \frac{||T||^k}{|\lambda|^k} < \infty 
	\end{align*}
	and so it makes since to define an operator
	$$
		S_\lambda := \sum_{k=0}^\infty \frac{T^k}{\lambda^k}
	$$
	where  $ S_\lambda \in B(\Hlbt) $.
	
	Now clearly we can write
	$$
		(T-\lambda I) S_\lambda = (T + \frac{T^2}{\lambda} + \frac{T^3}{\lambda^2} + \cdots ) - (\lambda I + T + \frac{T^2}{\lambda} + \frac{T^3}{\lambda^2} + \cdots )
	$$
	and therefore $ (T-\lambda I) S_\lambda = - \lambda I $.
	Similarly $ S_\lambda (T-\lambda I) = - \lambda I $.
	That means that $ \lambda \in \rho(T) $, because we have found an inverse of $ (T-\lambda I ) $, namely
	$$ R_\lambda(T) = \frac{-S_\lambda}{\lambda}. $$
\end{proof}

Due to the way that $ \sigma(T) $ is defined it would be appropriate to prove that $ \rho(T) $ is open
in order to show that $ \sigma(T) $ is closed.
\begin{lemma}[$ \sigma(T) $ is closed]
\label{lem:sigma_closed}
	If $ \lambda_0 \in \rho(T) $ and $ a = ||R_{\lambda_0}(T)||^{-1} > 0 $ then 
	$ \{ \lambda \in \Cplx : | \lambda - \lambda_0 | < a \} \subseteq \rho(T) $. 
\end{lemma}
What this essentially says is that we can always cook up an open $a$-ball around any $ \lambda_0 \in \rho(T) $.
\begin{proof}
	Let $ \lambda \in \Cplx $ be such that $ |\lambda - \lambda_0 | < a $.
	We can see that since $$ |\lambda - \lambda_0| \cdot || R_{\lambda_0} (T) || < 1 $$ then 
	$$
		\sum_{k=0}^\infty |\lambda - \lambda_0|^k\cdot || R_{\lambda_0}(T) ||^k < \infty
	$$
	and so it makes since to define an operator
	$$
		S_\lambda := \sum_{k=0}^\infty (\lambda -\lambda_0)^k R_{\lambda_0}(T)^{k}
	$$
	where $ S_\lambda \in B(\Hlbt) $.
	Now by writing $ \mu = \lambda - \lambda_0 $ we have that
	\begin{align*}
		(T - \lambda)S_\lambda &= (T - \lambda_0 I - \mu I)S_\lambda \\
			&= \left( (T- \lambda_0 I) + \mu I + \mu^2 R_{\lambda_0}(T) + \mu^3R_{\lambda_0}(T) + \cdots \right) \\
			& \ \ \ - \left( \mu I + \mu^2 R_{\lambda_0}(T) + \mu^3 R_{\lambda_0}(T)^2 + \cdots  \right) \\
			&= (T - \lambda_0 I ).
	\end{align*}
	In exactly the same manner we see that $ S_\lambda(T-\lambda I) = (T - \lambda_0 I)$.
	
	Now since $ (T-\lambda_0 I ) $ is invertable, with inverse $ R_{\lambda_0} $ it follows that 
	$$
		R_{\lambda_0}S\lambda (T -\lambda I ) = I
	$$
	and so $ \lambda \in \rho(T) $ and in particular $ R_{\lambda} = R_{\lambda_0}S_{\lambda} $.
	This shows that $ \rho(T) $ is open and hence $ \sigma(T) $ is closed.
\end{proof}

\begin{lemma}[$ \sigma(T) $ is non-empty]
\end{lemma}
\begin{proof}
	We assume $ \sigma(T) \neq \emptyset $ and arrive at a contradiction.
	
	Define a function $$ f_{T}(z) := R_{z}(T). $$
	By lemma \ref{lem:sigma_closed} for any $ \lambda_0 \in \Cplx $ we can write
	$$ \sum_{k=0}^\infty (z-\lambda_0)^k R_{\lambda_0}(T)^{k+1}. $$
	We can also further say that this series converges for all $ z \in \Cplx $.
	This means $ f $ is an entire function.
	By using the ideas of lemma \ref{lem:sigma_bdd} we can also write 
	$$
		f_T(z) = R_{z}(T) = - \frac{1}{z} \sum_{k=0}^\infty \frac{T^k}{z^k}
	$$
	and so $ f(z) $ is bounded.
	Now by Liouville's theorem $ f \equiv C $ where  $ C $ is a constant, but this is clearly a contradiction.
\end{proof}

\section*{Spectral Radius}

In this section we define the spectral radius of an operator and show a result that relates the norm of the operator
to the spectral radius. We then go on to show that if $ T $ is self adjoint then this relationship can be further simplified.

\begin{definition}[Spectral Radius]
	For any $ T \in B(\Hlbt) $ define $$ r(t) := \sup \{ |\lambda| : \lambda \in \sigma(T) \} $$.
\end{definition}

\begin{theorem}
	\label{thm:spec_rad}
	For any $ T \in B(\Hlbt) $ we have that 
	$$
		r(T) = \lim_{n \lra \infty} ||T^n||^\frac{1}{n}
	$$
\end{theorem}
We use several lemmas to prove this result.
Define $ \alpha(T) = \lim_{n \lra \infty}  ||T^n ||^\frac{1}{n} $.
\begin{lemma}
	$ \alpha(T) $ is well defined.
\end{lemma}
\begin{proof}
	Let $ \alpha_n = ||T^n|| $ and observe that by Cauchy-Schwartz
	\begin{align*}
		||T^{n+k} || \leq ||T^n|| \cdot ||T^k||.	& & \circledast
	\end{align*}
	Let $$ \alpha = \inf_{n \geq 1} \alpha^\frac{1}{n}_n $$ and fix any $ \epsilon > 0 $.
	
	Pick an $ m \in \Ntrl $ such that $ \alpha\leq \alpha_m^\frac{1}{m}< \alpha + \epsilon $.
	
	For any $ n \in \Ntrl $ we can pick $ k, b \in \Ntrl $ such that $ n = km + b $ and $ b < m $.
	
	Then by using the observation from $ \circledast $ we can write 
	\begin{align*}
		\alpha &\leq \alpha_{n}^\frac{1}{n} \\
			&\leq \alpha_{km}^\frac{1}{n} \cdot \alpha_{b}^\frac{1}{n} \\
			&\leq \alpha_{m}^\frac{k}{n} \cdot \alpha_{b}^\frac{1}{n} \\
			&= \left(\alpha_m^\frac{1}{m} \right)^\frac{km}{n} \cdot \alpha_b^\frac{1}{n} \\
			&\leq (\alpha + \epsilon)^{1 - \frac{b}{n}} \cdot \left( \max_{1 \leq b \leq m} a_b\right)^\frac{1}{n}.
	\end{align*}
	By letting $ n \lra \infty $ then we can pinch $ \alpha_{n}^\frac{1}{n} $ between $ \alpha $ and $ \alpha + \epsilon $,
	i.e. 
	$$
		\alpha \leq \liminf_{n \lra \infty} \alpha_n^\frac{1}{n} \leq \limsup_{n \lra \infty} \alpha_n^\frac{1}{n} \leq \alpha + \epsilon.
	$$
	So we can say that $ \alpha(T) = \alpha $.
\end{proof}
\begin{lemma}
	\label{lem:lambda_alpha_rho}
	If $ |\lambda| > \alpha(T) $ then $ \lambda \in \rho(T) $
\end{lemma}
\begin{proof}
	Note that
	if $ | \lambda | > \alpha(T) $ then there exists some $ N \in \Ntrl $ such that for any $ n \geq N $ we have that
	$$ 
		|| T^n ||\frac{1}{n} < \alpha(T).
	$$
	If we let $ q = \frac{\alpha(T)}{|\lambda|} $  then we can re-write this as
	$$
		||T^n|| < q^n |\lambda|^n 
	$$
	and so $ \frac{||T^n||}{|\lambda|^n} < q^n $.
	By noting that $ 0 < q < 1 $ we can say that the operator defined by
	$$ S_\lambda = - \sum_{n=0}^\infty \frac{T^n}{\lambda^n} $$
	converges absoluetly.
	
	Now since $ (T-\lambda I) S_\lambda = S_\lambda (T - \lambda I) = I $ (by a result similar to that in lemma \ref{lem:sigma_bdd})
	we can say that $ \lambda \in \rho(T) $.
\end{proof}

\begin{lemma}
	\label{lem:lambda_alpha_r}
	If $ |\lambda| > \alpha(T) $ then $ |\lambda| \geq r(T) $.
\end{lemma}
\begin{proof}
	If $ |\lambda| > \alpha(T) $ and $ |\lambda| < r(T) $ then because of the way we define $ r $ there must exist a $ \lambda_1 \in \rho(T) $ such that
	$ \alpha(T) < |\lambda| < |\lambda_1| < r(T) $. This is a contradition because lemma \ref{lem:lambda_alpha_rho} shows
	$ \lambda_0 \in \rho(T) $.
\end{proof}

What \ref{lem:lambda_alpha_r} says is that $ \alpha(T) \geq r(T) $.
We can now prove a final result which will amount to a proof of theorem \ref{thm:spec_rad}

\begin{proof}[of theorem \ref{thm:spec_rad}]
Assume, for a contradiction that $ \alpha(T) > r(T) $. Then we can pick some $ \lambda \in \Cplx $ such that 
$$
	\alpha(T) < |\lambda| < r(T).
$$
Fix some $ x, y \in \Hlbt $ and define a function
$$
	f_T(z) := R_z(T) = -\frac{1}{z} \sum_{k=0}^\infty \frac{\langle T^n x, y \rangle}{z^n}.
$$
This function is defined if $ |z| > \alpha(T) $ (by applying the same idea as in lemma \ref{lem:lambda_alpha_rho}).

Since $ \sigma(T) \subseteq \{ |z| \leq r(T) \} $ and so $ f $ is holomorphic for all $ |z| > r(T) $ so
it admits a power series expansion on the whole disc. It follows that the expansion given above for the wider disc
$$ \{ |z| > r(T) \} $$

This means that the series 
$$
	\sum_{n=0}^\infty \frac{\langle T^n x, y \rangle}{\lambda^n}
$$
converges.

This means that we must have the tail terms of this series going to zero, i.e.
$$
	\lim_{n \lra \infty} \frac{|\langle T^n x, y \rangle|}{|\lambda|^n} = 0
$$
and so by the uniform boundedness principle we must have a $ c > 0 $ such that
$$
	||T^n|| \leq c|\lambda|^n
$$
for all $ n $ and so
$$
	||T^n||^\frac{1}{n} \leq c^\frac{1}{n} |\lambda|
$$
which leads to a contradiction because it implies that $ \alpha(t) = \lim_{n \lra \infty} ||T^n||^\frac{1}{n} \leq |\lambda| $.
\end{proof}

We now consider the case of a self-adjoint bounded linear operator. In this case a formula formula for $ r(T) $
is rather much more simple.
\begin{lemma}
	If $ T \in B(\Hlbt) $ and $ T = T^* $ then $ r(T) = ||T|| $.
\end{lemma}
\begin{proof}
	From the $C^*$ identity we have that $ ||T||^2 = ||T^*T|| = ||T^2|| $.
	
	We can generalise this to say $ ||T||^{2n} = ||T^{2n}|| $ for all $ n \in \Ntrl $.
	By noting that we can write
	$$
		r(T) = \lim_{n \lra \infty} ||T^n||\frac{1}{n} = \lim_{n \lra \infty}||T^{2^n}||^{2^{-n}} = ||T||
	$$
	we have the result.
\end{proof}

\section*{Rational Calculus}

\end{document}
