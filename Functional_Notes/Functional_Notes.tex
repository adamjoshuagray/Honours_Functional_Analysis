\documentclass{unswmaths}
\usepackage{unswshortcuts}
\begin{document}
\author{Adam J. Gray}
\title{Lecture Notes}
\subject{Functional Analysis}
\studentno{3329798}

\unswtitle

\section*{Hilbert Spaces}

\begin{definition}
	A Hilbert space $ \Hlbt $ is a real or complex inner product space that is also a complete
	metric space with respect to the metric induced by the inner product. 
	\footnote{This definition is more or less taken straight from Wikipedia (3 June 2014).}
\end{definition}
	We say that the inner product is a function $ \langle \cdot , \cdot \rangle : \Hlbt \times \Hlbt \lra \Cplx $ (or $ \Rl $).
	which satisfies the following properties:
	\begin{itemize}
		\item $ \langle y, x \rangle = \overline{\langle x, y \rangle }$ for $ x, y \in \Hlbt $.
		\item $ \langle \alpha x_1 + \beta x_2, y \rangle = \alpha \langle x_1, y \rangle + \beta \langle x_2, y \rangle $ for $ x_1, x_2, y \in \Hlbt $ and $ \alpha, \beta \in \Cplx $.
		\item $ \langle x, x \rangle \geq 0 $ for all $ x \in \Hlbt $. Moreover $ \langle x, x \rangle = 0 $ if and only if $ x = 0 $.
	\end{itemize}
	Note that from this inner product we can easily define  a norm by writing 
	$$ ||x|| = \sqrt{\langle x, x \rangle}. $$
	
	We will now look at a couple of examples of Hilbert spaces.
	
\subsection*{ $ \Cplx $ and $ \Rl $} 
	Without being tedious checking that all the requirements are fulfilled we will simply state that both $ \Cplx $ and $ \Rl $
	are Hilbert spaces.

\subsection*{ $ \ell^2 $ }
	We will show that many of the properties of a Hilbert space are fulfilled by $ \ell^2 $ and we will
	claim that $ \ell^2 $ is infect a Hilbert space. To do this will will first define $ \ell^2 $ and then
	check that the inner product axioms are satisfied.
	
\begin{definition}[ $\ell^2 $ ]
	$$ 
		\ell^2 = \left\{ (x_1, x_2, x_3, \ldots ) : x_k \in \Cplx, \sum_{k=1}^\infty |x_k|^2 \right\}
	$$
	$$
		\langle \mathbf{x}, \mathbf{y} \rangle =  \sum_{k=1}^\infty x_k \overline{y}_k 
	$$
\end{definition}

Firstly we show that all of the inner product axioms are satisfied for the inner product here.

For $ \mathbf{x}, \mathbf{y} \in \ell^2 $ see that
\begin{align*}
	\langle \mathbf{y}, \mathbf{x} \rangle &= \sum_{k=1}^\infty y_k \overline{x}_{k} \\
		&= \overline{\sum_{k=1}^\infty \overline{y}_k x_{k}}\\
		&= \overline{ \langle \mathbf{x}, \mathbf{y} \rangle }.
\end{align*}

For  $ \mathbf{x}, \mathbf{y}, \mathbf{z} \in \ell^2 $ and $ \alpha, \beta \in \Cplx $
\begin{align*}
	\langle \alpha \mathbf{x} + \beta \mathbf{y}, \mathbf{z} \rangle &= \sum_{k = 1}^\infty (\alpha x_k + \beta y_k) \overline{z}_k  \\
		&= \alpha \sum_{k=1}^\infty x_k \overline{z}_k + \beta \sum_{k=1}^\infty y_k \overline{z}_k \\
		&= \alpha \langle \mathbf{x}, \mathbf{z} \rangle + \beta \langle \mathbf{y}, \mathbf{z} \rangle.
\end{align*}

For $ \mathbf{x} \in \ell^2 $ see that

\begin{align*}
	\langle \mathbf{x}, \mathbf{x} \rangle &= \sum_{k=1}^\infty x_k \overline{x}_k \\
		&= \sum_{k =1}^\infty |x_k|^2 \\
		&\geq 0
\end{align*}
and in particular it is clear to see that $ \langle \mathbf{x}, \mathbf{x} \rangle = 0 $ if and only if
$ \mathbf{x} = 0 $.

We now wish to show that for two elements $ \mathbf{x}, \mathbf{y} \in \ell^2 $ that the inner product does
in fact map to a complex number (and not just map to infinity).

For $ \mathbf{x}, \mathbf{y} \in \ell^2 $  see that
\begin{align*}
	|\langle \mathbf{x}, \mathbf{y} \rangle | &\leq \sum_{k=1}^\infty |x_k|\cdot|y_k| \\
\end{align*}
and by geometric - arithmetic mean inequality
\begin{align*}
	\sum_{k=1}^\infty |x_k|\cdot|y_k|
		&\leq \frac{1}{2} \underbrace{\sum_{k=1}^\infty |x_k|^2}_{ < \infty \text{ because } \mathbf{x} \in \ell^2 } 
			+ \frac{1}{2} \underbrace{\sum_{k=1}^\infty |y_k|^2}_{< \infty \text{ because } \mathbf{x} \in \ell^2} \\
			&< \infty.
\end{align*}

We now show that $ \ell^2 $ is in fact closed under addition, and hence a linear space.
For $ \mathbf{x}, \mathbf{y} \in \ell^2 $ we see that by finite dimension Cauchy-Schwartz 

\begin{align*}
	\sum_{k=1}^\infty |x_k + y_k|  \leq \underbrace{\sum_{k=1}^N |x_k|^2}_{< \infty}  + \underbrace{\sum_{k=1}^N |y_k|}_{< \infty}
\end{align*}
and by letting $ N \lra \infty $ we see that $ || \mathbf{x} + \mathbf{y} || < \infty $ and hence $ \mathbf{x} + \mathbf{y} \in \ell^2 $.

While we have not shown that $ \ell^2 $ is complete we do claim that $ \ell^2 $ is complete and hence a Hilbert space.

\subsection*{$ C[-1, 1] $}

The aim with $ C[-1, 1] $ is to show that with the usual inner product, $ C[-1, 1] $ is not a Hilbert space, because it is 
not complete.

For clarity we define the inner product on $ C[-1,1] $, for $ f, g \in C[-1,1] $ by 
$$
	\langle f, g \rangle = \int_{-1}^1 f(t) \overline{g(t)} dt.
$$

Do show that $ C[-1, 1] $ is not complete define the sequence of function $ f_n $ by
\begin{align*}
	f_n(t) = 
	\begin{cases}
		0	& \text{ if } t \not\in \left[\frac{-1}{n}, \frac{1}{n} \right] \\
		\sqrt{n} & \text{ if } x \in \left[ \frac{-1}{2n}, \frac{1}{2n} \right] \\
		\text{linear} & \text{elsewhere} 
	\end{cases}
\end{align*}
Now see that for each $ n $, $ f_n \in C[-1,1] $. It is also easy to see that for all $ n $, $ || f_n || \leq \sqrt{2} $.

Now consider the function $ F $ defined by 
$$
	F(x) = \sum_{n=1}^\infty \frac{1}{n^\frac{5}{4}} f_n(x).
$$
Observe that $$ ||F||  \leq \sqrt{2} \sum_{n=1}^\infty \frac{1}{n^\frac{5}{4}} < \infty $$
but see that 
$$
	F(0) = \sum_{n=1}^\infty \frac{1}{n^\frac{3}{4}} \lra \infty.
$$
This means that $ F \not\in C[-1,1] $ and hence $ C[-1,1] $ is not complete and therefore not a Hilbert space.

\subsection*{Completion Procedure for Hilbert Spaces}

Let $( \Hlbt_0, \langle \cdot, \cdot \rangle )$ be a pre-Hilbert space (i.e. not complete). 
We have the usual norm induced by the inner product, that is $ ||x|| = \sqrt{ \langle x, x \rangle } $.

To ``complete'' this as forma Hilbert space define the set of Cauchy Sequences
$$
	H = \{ X = (x_k)_{k=1}^\infty : x_k \in \Hlbt_0 \}.
$$

We then claim that $$ \Hlbt = \{ \overline{X} : X \in H \} $$ is a Hilbert space where we define
$$ \overline{X} = \{ Y \in H : Y \sim X \}. $$ We say that $$ X \sim Y $$ when $ \lim_{k \lra \infty} ||x_k - y_k|| = 0 $.

We define the inner product on this new Hilbert space $ \Hlbt $ by 
$$
	\langle \overline{X}, \overline{Y} \rangle = \lim_{k \lra \infty} \langle x_k, y_k \rangle. 
$$

Simple verification shows that is is infact a ``valid'' inner product.

We also clarify addition an scalar multiplcation in this new Hilbert space by saying that
\begin{align*}
	\overline{X} + \overline{Y} &= \overline{(x_k + y_k)} \\
	\alpha X = (\alpha x_k ).
\end{align*}


Since we have defined al lthe basic properties of this new Hilbert space, $ \Hlbt $, we check the ``correctness''
of this definition. 

Suppose we have
\begin{align*}
	X' = (x'_k) &\text{ and } X = (x_k) \\
	Y' = (y'_k) &\text{ and } Y = (y_k)
\end{align*}
such that
\begin{align*}
	X' \sim X \text{ and } Y' \sim Y
\end{align*}
then we wish to verify that
\begin{align*}
	X + Y \sim X' + Y' 
\end{align*}
and to do this see that
\begin{align*}
	\lim_{k \lra \infty} x_k + y_k &= \lim_{k \lra \infty} x'_k + y'_k.
\end{align*}

Also observe that
$$
	\alpha X \sim \alpha X'
$$
by the same reasoning.

To verify correctness we also need to show that the limit $ \lim_{k \lra \infty} \langle x_k , y_k \rangle $ exists
and that
$$
	\lim_{k \lra \infty} \langle x_k y_k \rangle = \lim_{ k \lra \infty } \langle x'_k, y'_k \rangle.
$$

Firstly see that 
$$
	\lim_{k \lra \infty} \langle x_k, y_k \rangle \leq \lim_{k \lra \infty} ||x_k||\cdot||y_k||
$$
by the Cuachy-Schwartz inequality and so the limit exists.

See that
$$
	\lim_{k \lra \infty} \langle x_k - x'_k, y_k - y'_k \rangle = \langle 0, 0 \rangle = 0 
$$
and so the desired equality holds.

To make this whole exercise purposful we need to check that the space we have defined is infact complete, and
therefore a Hilbert space.

Let $ \overline{X}_n \in \Hlbt $ be a sequence of elements of $ \Hlbt $ such that
$$
	\sum_{k \lra \infty} ||\overline{X}_k|| < \infty.
$$

Let $ \overline{X} = \sum_{k \lra \infty} \overline{X}_k $ and note that
by the definition of $ \Hlbt $ we must that $ \overline{X} \in \Hlbt $ and thus $\Hlbt $ is complete.

\subsection*{Important Identities and Inequalities}

\begin{theorem}[Cauchy-Schwartz]
	If $ \Hlbt $ is a Hilbert space and $ x, y \in \Hlbt $ then 
	$$
		| \langle x, y \rangle | \leq ||x||\cdot||y||.
	$$
\end{theorem}
\begin{proof}
	Define a function $ f : \Rl \lra \Rl $ as
	$$
		f(t) = ||x+ty||^2
	$$
	and note that $ f(t) \geq 0 $ for all $ t \in \Rl $.
	See that
	$$
		f(t) = ||x||^2 + t^2||y||^2 + 2t \Re{ \langle x,y \rangle } \geq 0 
	$$
	and since $ f(t) $ is a quadratic in $ t $ we can say that
	$$
		|\Re\langle x,y\rangle| \leq ||x|| \cdot ||y||
	$$
	by considering the discriminant.
	
	We wish to ``get rid of'' the $ \Re $ in this result.
	
	Now if $ \alpha = \langle x, y \rangle $ with $ \alpha \in \Cplx $ then we can define
	$ \operatorname{sgn}(\alpha) = \frac{\overline{\alpha}}{|\alpha|} $.
	
	Let $ \tilde{x} = \operatorname{sgn}(\alpha) \cdot x $ and from the already established result we can say that
	$$
		| \Re \langle \tilde{x}, y \rangle | \leq ||\tilde{x}||\cdot ||y||
	$$
	and
	\begin{align*}
		\Re \langle \tilde{x}, y \rangle &= \Re(\operatorname{sgn}(\alpha)\cdot \langle x, y \rangle ) \\
		&= |\langle x, y \rangle |.
	\end{align*}
	Also $$ ||\tilde{x}|| = | \operatorname{\alpha}| \cdot ||x|| = ||x|| $$
	and so we can remove the $ \Re $ part of the result by saying 
	\begin{align*}
		|\langle x, y \rangle | = | \Re \langle \tilde{x}, y \rangle | \leq ||\tilde{x}|| \cdot ||y|| = ||x||\cdot ||y||.
	\end{align*}
\end{proof}

\begin{theorem}[Triangle Inequality]
	If $ \Hlbt $ is a Hilbert space and $ x, y \in \Hlbt $ then
	$$
		||x + y|| \leq ||x|| + ||y||.
	$$
\end{theorem}
\begin{proof}
	See that
	\begin{align*}
		||x + y||^2 &= \langle x + y, x +y \rangle \\
			&= ||x||^2 + ||y||^2 + 2 \times \underbrace{\Re{\langle x, y \rangle}}_{ \leq ||x||\cdot||y|| \text{ by Cauchy-Schwartz} } \\
			&\leq ||x||^2 + ||y||^2 + 2||x||\cdot||y|| \\
			&= (||x|| + ||y||)^2
	\end{align*}
	and so
	$$
		||x + y|| \leq ||x|| + ||y||.
	$$
\end{proof}

\begin{theorem}[Parallelogram Law]
If $ \Hlbt $ is a Hilbert space and $ x, y \in \Hlbt $ then
\begin{align*}
	2||x||^2 + 2||y||^2 = ||x+y||^2 + ||x-y||^2.
\end{align*}
\end{theorem}
\begin{proof}
	See that
	\begin{align*}
		||x+y||^2 + ||x-y||^2 &= \langle x + y, x+y \rangle + \langle x - y, x -y \rangle \\
			&= |x||^2 + ||y||^2 + ||x||^2 + ||y||^2 \\
			& \ \ \ + \langle x, y \rangle + \langle y, x \rangle - \langle x, y \rangle - \langle y, x \rangle \\
			&= 2||x||^2 + 2||y||^2.
	\end{align*}
\end{proof}

\begin{theorem}[Polarization Identity]
	If $ \Hlbt $ is a complex Hilbert space and $ x, y \in \Hlbt $ then
	$$
		\langle x, y \rangle = \frac{1}{2} \left( ||x+y||^2 - ||x-y||^2 + i||x+iy||^2 - i||x-iy||^2 \right).
	$$
	If $ \Hlbt $ is a real Hilbert space then this can be reduced to 
	$$
		\langle x, y \rangle = \frac{1}{4} \left( ||x+y||^2 - ||x-y||^2 \right).
	$$
\end{theorem}
The proof of these facts follows in much that same way as for the parallelogram law.

\subsection*{Orthonormal Basis in Hilbert Spaces}

An orthonormal basis is a family of elements of $ \Hlbt $ which satisfy the following conditions
\begin{itemize}
	\item Mutual orthogonality \\
	\item Normality \\
	\item Completeness.
\end{itemize}

We wish to develop a ``Pythagorean Theorem'' for Hilbert spaces.

\section*{Spectral Theory}

Spectral theory is analogous to eigen-value theory for matrices.
In this case we look at spectral theory on bounded linear operators on 
a Hilbert space $ \Hlbt $. 

We define the following

For $ T \in B( \Hlbt ) $ we define the following
\begin{definition}[Resolvent Set]
	Define $ \rho(T) := \{ \lambda \in \Cplx : \exists (T-\lambda)^{-1}  \in B( \Hlbt ) \} $.
\end{definition}
\begin{definition}[Resolvent]
	Define $ R_\lambda(T) := (T - \lambda)^{-1} $
\end{definition}
\begin{definition}[Spectrum]
	Define $ \sigma(T) := \Cplx \setminus \rho(T) $.
\end{definition}

On could heuristically think of $ \lambda $ as being an eigen-value and we formalize this idea as follows.

If $ \Hlbt = \Cplx^n $ and $ T = T_A $ where $ A = \left( a_{j,k} \right)_{j,k=1}^n $ (a matrix) then 
if $ (T - \lambda I)^{-1} $ does not exist, i.e. $ \ker( T - \lambda I) \neq \mathbf{0} $ then $ \lambda $ is
an eigen-value.

We would like to formalize and prove the following three ideas;
\begin{itemize}
	\item $ \sigma(T) $ is bounded,
	\item $ \sigma(T) $ is closed, and
	\item $ \sigma(T) $ is non-empty.
\end{itemize}

\begin{lemma}[$\sigma(T) $ is bounded]
\label{lem:sigma_bdd}
	If $ | \lambda | > ||T|| $ then $ \lambda \in \rho(T) $.
\end{lemma}
\begin{proof}
	If we have that $ | \lambda | < ||T|| $ then 
	\begin{align*}
		\sum_{k=0}^\infty \frac{||T||^k}{|\lambda|^k} < \infty 
	\end{align*}
	and so it makes since to define an operator
	$$
		S_\lambda := \sum_{k=0}^\infty \frac{T^k}{\lambda^k}
	$$
	where  $ S_\lambda \in B(\Hlbt) $.
	
	Now clearly we can write
	$$
		(T-\lambda I) S_\lambda = (T + \frac{T^2}{\lambda} + \frac{T^3}{\lambda^2} + \cdots ) - (\lambda I + T + \frac{T^2}{\lambda} + \frac{T^3}{\lambda^2} + \cdots )
	$$
	and therefore $ (T-\lambda I) S_\lambda = - \lambda I $.
	Similarly $ S_\lambda (T-\lambda I) = - \lambda I $.
	That means that $ \lambda \in \rho(T) $, because we have found an inverse of $ (T-\lambda I ) $, namely
	$$ R_\lambda(T) = \frac{-S_\lambda}{\lambda}. $$
\end{proof}

Due to the way that $ \sigma(T) $ is defined it would be appropriate to prove that $ \rho(T) $ is open
in order to show that $ \sigma(T) $ is closed.
\begin{lemma}[$ \sigma(T) $ is closed]
\label{lem:sigma_closed}
	If $ \lambda_0 \in \rho(T) $ and $ a = ||R_{\lambda_0}(T)||^{-1} > 0 $ then 
	$ \{ \lambda \in \Cplx : | \lambda - \lambda_0 | < a \} \subseteq \rho(T) $. 
\end{lemma}
What this essentially says is that we can always cook up an open $a$-ball around any $ \lambda_0 \in \rho(T) $.
\begin{proof}
	Let $ \lambda \in \Cplx $ be such that $ |\lambda - \lambda_0 | < a $.
	We can see that since $$ |\lambda - \lambda_0| \cdot || R_{\lambda_0} (T) || < 1 $$ then 
	$$
		\sum_{k=0}^\infty |\lambda - \lambda_0|^k\cdot || R_{\lambda_0}(T) ||^k < \infty
	$$
	and so it makes since to define an operator
	$$
		S_\lambda := \sum_{k=0}^\infty (\lambda -\lambda_0)^k R_{\lambda_0}(T)^{k}
	$$
	where $ S_\lambda \in B(\Hlbt) $.
	Now by writing $ \mu = \lambda - \lambda_0 $ we have that
	\begin{align*}
		(T - \lambda)S_\lambda &= (T - \lambda_0 I - \mu I)S_\lambda \\
			&= \left( (T- \lambda_0 I) + \mu I + \mu^2 R_{\lambda_0}(T) + \mu^3R_{\lambda_0}(T) + \cdots \right) \\
			& \ \ \ - \left( \mu I + \mu^2 R_{\lambda_0}(T) + \mu^3 R_{\lambda_0}(T)^2 + \cdots  \right) \\
			&= (T - \lambda_0 I ).
	\end{align*}
	In exactly the same manner we see that $ S_\lambda(T-\lambda I) = (T - \lambda_0 I)$.
	
	Now since $ (T-\lambda_0 I ) $ is invertible, with inverse $ R_{\lambda_0} $ it follows that 
	$$
		R_{\lambda_0}S\lambda (T -\lambda I ) = I
	$$
	and so $ \lambda \in \rho(T) $ and in particular $ R_{\lambda} = R_{\lambda_0}S_{\lambda} $.
	This shows that $ \rho(T) $ is open and hence $ \sigma(T) $ is closed.
\end{proof}

\begin{lemma}[$ \sigma(T) $ is non-empty]
\end{lemma}
\begin{proof}
	We assume $ \sigma(T) \neq \emptyset $ and arrive at a contradiction.
	
	Define a function $$ f_{T}(z) := R_{z}(T). $$
	By lemma \ref{lem:sigma_closed} for any $ \lambda_0 \in \Cplx $ we can write
	$$ \sum_{k=0}^\infty (z-\lambda_0)^k R_{\lambda_0}(T)^{k+1}. $$
	We can also further say that this series converges for all $ z \in \Cplx $.
	This means $ f $ is an entire function.
	By using the ideas of lemma \ref{lem:sigma_bdd} we can also write 
	$$
		f_T(z) = R_{z}(T) = - \frac{1}{z} \sum_{k=0}^\infty \frac{T^k}{z^k}
	$$
	and so $ f(z) $ is bounded.
	Now by Liouville's theorem $ f \equiv C $ where  $ C $ is a constant, but this is clearly a contradiction.
\end{proof}

\section*{Spectral Radius}

In this section we define the spectral radius of an operator and show a result that relates the norm of the operator
to the spectral radius. We then go on to show that if $ T $ is self adjoint then this relationship can be further simplified.

\begin{definition}[Spectral Radius]
	For any $ T \in B(\Hlbt) $ define $$ r(t) := \sup \{ |\lambda| : \lambda \in \sigma(T) \} $$.
\end{definition}

\begin{theorem}
	\label{thm:spec_rad}
	For any $ T \in B(\Hlbt) $ we have that 
	$$
		r(T) = \lim_{n \lra \infty} ||T^n||^\frac{1}{n}
	$$
\end{theorem}
We use several lemmas to prove this result.
Define $ \alpha(T) = \lim_{n \lra \infty}  ||T^n ||^\frac{1}{n} $.
\begin{lemma}
	$ \alpha(T) $ is well defined.
\end{lemma}
\begin{proof}
	Let $ \alpha_n = ||T^n|| $ and observe that by Cauchy-Schwartz
	\begin{align*}
		||T^{n+k} || \leq ||T^n|| \cdot ||T^k||.	& & \circledast
	\end{align*}
	Let $$ \alpha = \inf_{n \geq 1} \alpha^\frac{1}{n}_n $$ and fix any $ \epsilon > 0 $.
	
	Pick an $ m \in \Ntrl $ such that $ \alpha\leq \alpha_m^\frac{1}{m}< \alpha + \epsilon $.
	
	For any $ n \in \Ntrl $ we can pick $ k, b \in \Ntrl $ such that $ n = km + b $ and $ b < m $.
	
	Then by using the observation from $ \circledast $ we can write 
	\begin{align*}
		\alpha &\leq \alpha_{n}^\frac{1}{n} \\
			&\leq \alpha_{km}^\frac{1}{n} \cdot \alpha_{b}^\frac{1}{n} \\
			&\leq \alpha_{m}^\frac{k}{n} \cdot \alpha_{b}^\frac{1}{n} \\
			&= \left(\alpha_m^\frac{1}{m} \right)^\frac{km}{n} \cdot \alpha_b^\frac{1}{n} \\
			&\leq (\alpha + \epsilon)^{1 - \frac{b}{n}} \cdot \left( \max_{1 \leq b \leq m} a_b\right)^\frac{1}{n}.
	\end{align*}
	By letting $ n \lra \infty $ then we can pinch $ \alpha_{n}^\frac{1}{n} $ between $ \alpha $ and $ \alpha + \epsilon $,
	i.e. 
	$$
		\alpha \leq \liminf_{n \lra \infty} \alpha_n^\frac{1}{n} \leq \limsup_{n \lra \infty} \alpha_n^\frac{1}{n} \leq \alpha + \epsilon.
	$$
	So we can say that $ \alpha(T) = \alpha $.
\end{proof}
\begin{lemma}
	\label{lem:lambda_alpha_rho}
	If $ |\lambda| > \alpha(T) $ then $ \lambda \in \rho(T) $
\end{lemma}
\begin{proof}
	Note that
	if $ | \lambda | > \alpha(T) $ then there exists some $ N \in \Ntrl $ such that for any $ n \geq N $ we have that
	$$ 
		|| T^n ||\frac{1}{n} < \alpha(T).
	$$
	If we let $ q = \frac{\alpha(T)}{|\lambda|} $  then we can re-write this as
	$$
		||T^n|| < q^n |\lambda|^n 
	$$
	and so $ \frac{||T^n||}{|\lambda|^n} < q^n $.
	By noting that $ 0 < q < 1 $ we can say that the operator defined by
	$$ S_\lambda = - \sum_{n=0}^\infty \frac{T^n}{\lambda^n} $$
	converges absolutely.
	
	Now since $ (T-\lambda I) S_\lambda = S_\lambda (T - \lambda I) = I $ (by a result similar to that in lemma \ref{lem:sigma_bdd})
	we can say that $ \lambda \in \rho(T) $.
\end{proof}

\begin{lemma}
	\label{lem:lambda_alpha_r}
	If $ |\lambda| > \alpha(T) $ then $ |\lambda| \geq r(T) $.
\end{lemma}
\begin{proof}
	If $ |\lambda| > \alpha(T) $ and $ |\lambda| < r(T) $ then because of the way we define $ r $ there must exist a $ \lambda_1 \in \rho(T) $ such that
	$ \alpha(T) < |\lambda| < |\lambda_1| < r(T) $. This is a contradiction because lemma \ref{lem:lambda_alpha_rho} shows
	$ \lambda_0 \in \rho(T) $.
\end{proof}

What \ref{lem:lambda_alpha_r} says is that $ \alpha(T) \geq r(T) $.
We can now prove a final result which will amount to a proof of theorem \ref{thm:spec_rad}

\begin{proof}[of theorem \ref{thm:spec_rad}]
Assume, for a contradiction that $ \alpha(T) > r(T) $. Then we can pick some $ \lambda \in \Cplx $ such that 
$$
	\alpha(T) < |\lambda| < r(T).
$$
Fix some $ x, y \in \Hlbt $ and define a function
$$
	f_T(z) := R_z(T) = -\frac{1}{z} \sum_{k=0}^\infty \frac{\langle T^n x, y \rangle}{z^n}.
$$
This function is defined if $ |z| > \alpha(T) $ (by applying the same idea as in lemma \ref{lem:lambda_alpha_rho}).

Since $ \sigma(T) \subseteq \{ |z| \leq r(T) \} $ and so $ f $ is holomorphic for all $ |z| > r(T) $ so
it admits a power series expansion on the whole disc. It follows that the expansion given above for the wider disc
$$ \{ |z| > r(T) \} $$

This means that the series 
$$
	\sum_{n=0}^\infty \frac{\langle T^n x, y \rangle}{\lambda^n}
$$
converges.

This means that we must have the tail terms of this series going to zero, i.e.
$$
	\lim_{n \lra \infty} \frac{|\langle T^n x, y \rangle|}{|\lambda|^n} = 0
$$
and so by the uniform boundedness principle we must have a $ c > 0 $ such that
$$
	||T^n|| \leq c|\lambda|^n
$$
for all $ n $ and so
$$
	||T^n||^\frac{1}{n} \leq c^\frac{1}{n} |\lambda|
$$
which leads to a contradiction because it implies that $ \alpha(t) = \lim_{n \lra \infty} ||T^n||^\frac{1}{n} \leq |\lambda| $.
\end{proof}

We now consider the case of a self-adjoint bounded linear operator. In this case a formula formula for $ r(T) $
is rather much more simple.
\begin{lemma}
	If $ T \in B(\Hlbt) $ and $ T = T^* $ then $ r(T) = ||T|| $.
\end{lemma}
\begin{proof}
	From the $C^*$ identity we have that $ ||T||^2 = ||T^*T|| = ||T^2|| $.
	
	We can generalize this to say $ ||T||^{2n} = ||T^{2n}|| $ for all $ n \in \Ntrl $.
	By noting that we can write
	$$
		r(T) = \lim_{n \lra \infty} ||T^n||\frac{1}{n} = \lim_{n \lra \infty}||T^{2^n}||^{2^{-n}} = ||T||
	$$
	we have the result.
\end{proof}

We now prove a collection of results about the spectrum of various types of operators.

\begin{theorem}
	If $ T \in B(\Hlbt) $ and $ T $ is self adjoint then $ \sigma(T) \subseteq \Rl $.
\end{theorem}
\begin{proof}
	Let $ \lambda \in \Cplx $ be such that $ \lambda = \alpha + i \beta $ with $ \beta \neq 0 $.
	Then 
	\begin{align*}
		||(T - \lambda I)x||^2 &= ||( T - \alpha I - i \beta I)x||^2 \\
			&= \langle (T - \alpha I - i \beta I)x, (T - \alpha I - i \beta I)x \rangle \\
			&= \langle (T - \alpha I + i \beta I) (T - \alpha I - i \beta I)x,x \rangle \\
			&= \langle \left((T - \alpha I)^2 + \beta I \right)x,x \rangle \\
			&= \langle (T - \alpha I)^2x + \beta Ix ,x \rangle \\
			&= \langle (T - \alpha I)^2 x, x \rangle + \beta^2 \langle x, x\rangle \\
			&= ||(T-\alpha I)x||^2 + \beta^2 ||x||^2.
	\end{align*}
	
	We can therefore write that $ || (T- \lambda I)x || \geq \Im(\lambda)||x|| $ which means that
	$ (T - \lambda I ) $ is injective.
	
	We can also see that $ T $ is subjective because
	$$
		\left(\operatorname{Im}(T - \lambda I)\right)^\perp = \ker(T - \overline{\lambda}) = \{ 0 \}.
	$$
	This means that $$ || (T - \lambda I)^{-1} || \leq \Im(\lambda)^{-1} $$ which is to say that 
	$ \lambda \in \rho(T) $.
	
	This means that $ \lambda \not\in \sigma(T) $ which means that any $ z \in \sigma(T) $ must be such that
	$ \Im(z) = 0 $.
\end{proof}

\section*{Rational Calculus}
Here we generalist ideas about functions on the real numbers to functions of bounded linear operators.

Firstly we consider a function 
$$
	f(z) = \frac{(z-\alpha_1) (z - \alpha_2) \cdots (z - \alpha_n)}{(z - \lambda_1) (z- \lambda_2) \cdots (z - \lambda_m)}
$$
and call this function a rational function. This is because it can be considered the quotient of two polynomials. (The complete factorization here being
permitted by the fundamental theorem of calculus).

From this definition of $ f $ is clear to see that the $ \alpha_j $ constitute the roots of $ f $ and the $ \lambda_j $ constitute the 
poles of $ f $.

Now for any $ T \in B(\Hlbt) $ and $ \lambda_1, \ldots , \lambda_m \not\in \sigma(T) $ we define a rational calculus by
$$
	f(T) := (T-\alpha_1 I)(T- \alpha_2 I) \cdots (T - \alpha_n I) R_{\lambda_1}(T)R_{\lambda_2}(T) \cdots R_{\lambda_m}(T)
$$

We now have a simple theorem we wish to consider which relates the spectrum of $ f(T) $ to the spectrum of $ T $.

\begin{theorem}
	For $ T \in B(\Hlbt) $ and with $ f $ defined as above (i.e. with $ \lambda_j \not\in \sigma(T) $) then
	$$
		\sigma(f(T)) = f(\sigma(T)). 
			\footnote{We define $ f(\sigma(T)) := \{ f(z) : z \in \sigma(T) \} $. That is to say $ f(\sigma(T)) $ returns a set with 
	$ f $ applied to each element of the set which was supplied as the argument.}
	$$
\end{theorem}
\begin{proof}
	This proof proceeds in two parts:
	\begin{itemize}
		\item	$f(\sigma(T)) \subseteq \sigma(f(T)) $
		
		Fix a $ \lambda \in \sigma(T) $ and let $ \mu = f(\lambda) $. Define a function $ g $ such that
		$$
			g(z) = f(z) - \mu.
		$$
		Now since $ g(\lambda) = 0  $, $ g $ admits a factorization $ g(z) = (z - \lambda)\frac{p(z)}{q(z)} $
		where $ p, q $ are polynomials. 
		
		Now consider
		$$
			g(T) = f(T) - \mu I = \underbrace{(T - \lambda I)}_{\circledast} p(T) q(T)^{-1} 
		$$
		and note that $ \circledast $ is not invertible because $ \lambda \in \sigma(T) $ and hence
		$$
			f(T) - \mu 
		$$
		is not invertible and so $ \mu \in \sigma(f(T)) $.
		\item $ \sigma(f(T)) \subseteq f(\sigma(T)) $
		
		Let $\mu \in \sigma(f(T)) $ and define a function 
		$$
			g(z) = f(z) - \mu = \frac{(z-\alpha_1)((z-\alpha_2)\cdots(z - \alpha_n)}{(z-\lambda_1)(z-\lambda_2)\cdots (z-\lambda_m)}
		$$
		so we can write
		$$
			f(T) - \mu I = \underbrace{(T - \alpha_1 I) (T-\alpha_2 I) \cdots (T - \alpha_n I)}_{\circledast \circledast} \underbrace{R_{\lambda_1}(T) R_{\lambda_2}(T) \cdots R_{\lambda_m}(T)}_{\circledast \circledast \circledast}.
		$$
		Note $ f(T) - \mu I $ is not invertible because $ \mu \in \sigma(f(T)) $. By definition everything in $ \circledast \circledast \circledast $ must be invertible
		so something in $ \circledast \circledast $ must not be invertible. This means that one of $ \alpha_1, \alpha_2 \ldots \alpha_n $ is in $ \sigma(T) $.
		
		Now for all $ k \in \{ 1, \ldots, n \} $ we have that
		$$
			g(\alpha_k) = 0 
		$$
		which would force $ f(\alpha_k) = \mu $ and so $ \mu \in f(\sigma(T)) $.
	\end{itemize}
\end{proof}

We have an immediate corollary in the case where $ T $ is self adjoint.

\begin{corollary}
	If $ T = T^* $ then $$||f(T)|| = \sup_{\lambda \in \sigma(T)} |f(\lambda)|$$
\end{corollary}
\begin{proof}
	By the $ C^* $ identity we have that
	\begin{align*}
		||f(T)||^2 &= ||f(T)^*f(T)|| \\
			&= \overline{f}(T^*)f(T) ||.
	\end{align*}
	Letting $ g(z) = \overline{f}(z)f(z) $ we have that since $ T $, and hence $ g(T) $, is self adjoint that  
	$$
		||g(T)|| = r(g(T)) = \sup_{z \in \sigma(T)} |g(z)| = \sup_{z \in \sigma(T)} |f(z)|^2.
	$$
	That is to say $$||f(T)|| = \sup_{\lambda \in \sigma(T)} |f(\lambda)|.$$
\end{proof}
\end{document}
